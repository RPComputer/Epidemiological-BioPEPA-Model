


%
There are four reaction channels in the model.  Reaction $r_0$
represents synthesis (of compound \textit{E:S}) at the 
constant rate $k_0$.  The other three reactions are the usual
Michaelis-Menten enzymatic reactions.
%
\begin{eqnarray*}
r_0 & = & [ k_0 ]\\%
r_1 & = & [ k_1 \times  E \times  S ]\\%
r_{-1} & = & [ k_{-1} \times  \hbox{\textit{E:S}} ]\\%
r_2 & = & [ k_2 \times  \hbox{\textit{E:S}} ]\\%
\end{eqnarray*}
%
Five species are involved in the reactions.  These are the
enzyme \textit{E}, the substrate \textit{S}, the compound
\textit{E:S}, the product \textit{P} and the catalyst
\textit{X}, which is needed to synthesize the compound
\textit{E:S}.
%
\begin{eqnarray*}
E & = & r_1{\downarrow} +  r_{-1}{\uparrow}  +  r_2{\uparrow} \\%
S & = & r_1{\downarrow}  +  r_{-1}{\uparrow} \\%
\hbox{\textit{E:S}} & = & r_0{\uparrow} + r_1{\uparrow}  +  r_{-1}{\downarrow}  +  r_2{\downarrow} \\%
P & = & r_2{\uparrow} \\%
X & = & (r_0, 1) \odot  X\\%
\end{eqnarray*}
%
The component \textit{T} does not represent a chemical species.  
It is a model component used to plot functions of the species
numbers.  In this case \textit{T} is merely the sum of the number of 
molecules of the species involved in the reactions.  The compound
\textit{E:S} is counted as two molecules.
%
\begin{eqnarray*}
T & = & [ E + S + (2 \times  \hbox{\textit{E:S}}) + P + X ]\\%
\end{eqnarray*}
%
The species are involved in the reactions as described 
in the \emph{model equation} below.
%
\begin{eqnarray*}
(E \sync{\{r_1, r_{-1}, r_2\}} (S \sync{\{r_1, r_{-1}\}} ((X \sync{\{r_0\}} \hbox{\textit{E:S}}) \sync{\{r_2\}} P)))%
\end{eqnarray*}
